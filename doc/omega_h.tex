\documentclass{article}

\usepackage{listings}

\title{The Omega\_h Users Manual}
\author{Dan Ibanez\\
Sandia National Laboratories\\
daibane@sandia.gov}

\begin{document}

\maketitle

\section{Obtaining}

Omega\_h is developed and distributed via GitHub,
a popular software hosting platform based on the
Git distributed version control system.

The most common way to obtain Omega\_h is to
use Git to clone the repository and automatically
check out the \texttt{master} branch:

\begin{lstlisting}[language=bash]
git clone git@github.com:ibaned/omega_h.git
\end{lstlisting}

\section{Compiling}

\subsection{Dependencies}

All code dependencies of Omega\_h are optional,
meaning that one can compile it and obtain a fairly
functional code for mesh adaptation, although
it will not have parallel features yet.
Optional dependencies of Omega\_h are:

\begin{enumerate}

\item Zlib: This widely used and installed C library implements
efficient data compression algorithms.
Omega\_h uses it compress its own `.osh` file format, as well
as to produce compressed `.vtu` files.

\item MPI: The Message Passing Interface is a standard
defining (at least) a C library that enables multi-process parallelism.
This is required if you want to use multi-process parallelism
in Omega\_h.
The two good open-source implementations that Omega\_h is known
to work with are MPICH and OpenMPI, and we strongly recommend
MPICH for its support of the latest MPI standard and its
proper handling of memory.

\item Kokkos: This C++11 library implements shared-memory parallelism
constructs and allows Omega\_h to (mostly) not worry about the details
of OpenMP and CUDA.
It is required if you want to use shared-memory parallelism in Omega\_h.
Kokkos can be obtained as part of Trilinos, and Trilinos can be configured
to compile and install only Kokkos.

\item libMeshb: This C library implements the `.mesh` and `.meshb`
file formats used by INRIA, NASA, and others.
It is required to read and write `.meshb` files from Omega\_h.
Note that currently Omega\_h follows a particular convention in what
those files are expected to contain, namely elements, vertices,
and sides on the boundary.

\item EGADS: This C API wraps over OpenCASCADE in a human-manageable way.
It is required if you want Omega\_h to snap new vertices to geometry.
Note that classification in Omega\_h should match the numbering
of geometric entities in EGADS.

\end{enumerate}

Each of these dependencies

\subsection{Options}

\subsection{Utility Programs}

\section{Integration}

\subsection{Header and Library}

\subsection{Using CMake}

\subsection{The Omega\_h Namespace}

\section{Read and Write: The Array Classes}

\section{The Mesh: A (Mostly) Immutable Cache}

\end{document}
